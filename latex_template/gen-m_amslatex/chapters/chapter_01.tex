%-----------------------------------------------------------------------
% Beginning of chapter_01.tex
%-----------------------------------------------------------------------
%
%  AMS-LaTeX sample file for a chapter of a monograph, to be used with
%  an AMS monograph document class.  This is a data file input by
%  chapter.tex.
%
%  Use this file as a model for a chapter; DO NOT START BY removing its
%  contents and filling in your own text.
% 
%%%%%%%%%%%%%%%%%%%%%%%%%%%%%%%%%%%%%%%%%%%%%%%%%%%%%%%%%%%%%%%%%%%%%%%%

% \part{This is a Part Title Sample}

\chapter{The Real and Complex Number Systems}\label{ch:real_complex_numbers}

\section*{Introduction}
A satisfactory discussion of the main concepts of analysis (such as convergence, continuity, differentiation, and integration) 
must be based on an accurately defined number concept. We shall not, however, enter into any discussion of the axioms that govern 
the arithmetic of the integers, but assume familiarity with the rational numbers (i.e., the numbers of the form 
$m/n$, where $m$ and $n$ are integers and $n \neq 0$).

\indent The rational number system is inadequate for many purposes, both as a field and as an ordered set. (These terms will be 
defined in Secs. \ref{some section} and \ref{some section}.) For instance, there is no rational $p$ such that $p^2 = 2$. (We shall prove this presently.) 
This leads to the introduction of so-called ``irrational numbers'' which are often written as infinite decimal expansions and are 
considered to be ``approximated'' by the corresponding finite decimals. Thus the sequence 

\begin{equation*}
1, 1.4, 1.41, 1.414, 1.4142,\cdots
\end{equation*}

``tends to $\sqrt{2}$.'' But unless the irrational number $\sqrt{2}$ has been clearly defined, the question must arise: Just what 
is it that this sequence ``tends to''?

\indent This sort of question can be answered as soon as the so-called ``real number system''
is constructed.

\begin{example}
  We now show that the equation
  \begin{equation}\label{eq:p^2=2}
    p^2 = 2
  \end{equation}
  is not satisfied by any rational $p$. If there were such a $p$, we could write $p=m/n$ where $m$ and $n$ are integers 
  that are not both even. Let us assume this is done. Then \ref{eq:p^2=2} implies
  \begin{equation}\label{eq:m^2-2n^2}
    m^2 =2n^2
  \end{equation}
  this shows that $m^2$ is even. Hence $m$ is even (if $m$ were odd, $m^2$ would be odd), and so $m^2$ is divisible by $4$. 
  It follows that the right side of \ref{eq:m^2-2n^2} is divisible by $4$. so that $n^2$ is even, which implies that $n$ is even.\\
  \indent The assumptions that \ref{eq:p^2=2} holds thus leads to the contradiction that both $m$ and $n$ are even, contrary to our 
  choice of $m$ and $n$. Hence \ref{eq:p^2=2} is impossible for rational $p$.





  \indent If $p$ is in $A$ then $p^2-2<0$, \ref{3} shows that $q>p$, and \ref{4} shows that $q^2<2$. Thus $q$ is in $A$.\\
  \indent If $p$ is in $B$ then $p^2-2>0$, \ref{3} shows that $0<q<p$, and \ref{4} shows that $q^2>2$. Thus $q$ is in $B$.
\end{example}

\begin{remark}
  The purpose of the above discussion has been to show that the rational number system has certain gaps, in spite of the 
  fact that between any two rationals there is another: If $r<s$ then $r<\frac{(r+s)}{2}<s$. The real number system fills these 
  gaps. This is the principal reason for the fundamental role which it plays in analysis.\\
  \indent In order to elucidate its structure, as well as that of the complex numbers, we start with a brief discussion of the 
  general concepts of \emph{ordered set} and \emph{field}. Here is some of the standard set-theoretic terminology that will be 
  used throughout this book.
\end{remark}


\begin{definition}
  If $A$ is any set (whose elements may be numbers or any other objects), we write $x \in A$ to indicate that x is a member 
  (or an element) of $A$.\newline
   If $x$ is not a member of $A$, we write: $x \notin A$.\\
  \indent The set which contains no element will be called the \emph{empty set}. If a set has at least one element, it is called 
  \emph{nonempty}.\\

\end{definition}









\begin{definition}
  Throughout Chap. \ref{ch:real_complex_numbers}, the set of all rational numbers will be denoted by $\mathbb{Q}$.
\end{definition}





\section{Ordered Sets}

\begin{definition}
  Let $S$ be a set. An \emph{order} on $S$ is a relation, denoted by $<$, with the following two properties:
  \begin{enumerate}[(i)]
    \item If $x \in S$ and $y \in S$ then one and only one of the statements 
    \begin{equation*}
    x < y,\quad x = y,\quad y < x  
    \end{equation*}
    is true.
    \item If $x,y,z \in S$, if $x<y$ and $y<x$, then $x<z$.
  \end{enumerate}
  The statement ``$x < y$'' may be read as ``$x$ is less than $y$'' or ``$x$ is smaller than $y$'' or ``$x$ precede $y$''.\\
    \indent It is often convenient to write $y > x$ in place of $x < y$.\\
    \indent The notation $x \leq y$ indicates that $x < y$ or $x = y$, without specifying which of these is to hold. In other words,
    $x \leq y$ is the negation of $x>y$
\end{definition}

\begin{definition}
  An \emph{ordered set} is a set $S$ in which an order is defined. For example, $\mathbb{Q}$ is an ordered set if 
  $r<s$ is defined to mean that $s-r$ is a positive rational number.
\end{definition}


\section*{Fields}
This is an example of an unnumbered first-level heading.
\section*{The Real Field}
This is an example of an unnumbered first-level heading.
\section*{The Extended Real Number System}
This is an example of an unnumbered first-level heading.
\section*{The Complex Field}
This is an example of an unnumbered first-level heading.
\section*{Euclidean Spaces}
This is an example of an unnumbered first-level heading.
\section*{Appendix}
This is an example of an unnumbered first-level heading.
\section*{Exercises}
\begin{xca}
  This is an example of the \texttt{xca} environment. This environment is
  used for exercises which occur within a section.
  \end{xca}





  \section{Some more list types}
This is an example of a bulleted list.

\begin{itemize}
\item $\mathcal{J}_g$ of dimension $3g-3$;
\item $\mathcal{E}^2_g=\{$Pryms of double covers of $C=\openbox$ with
normalization of $C$ hyperelliptic of genus $g-1\}$ of dimension $2g$;
\item $\mathcal{E}^2_{1,g-1}=\{$Pryms of double covers of
$C=\openbox^H_{P^1}$ with $H$ hyperelliptic of genus $g-2\}$ of
dimension $2g-1$;
\item $\mathcal{P}^2_{t,g-t}$ for $2\le t\le g/2=\{$Pryms of double
covers of $C=\openbox^{C'}_{C''}$ with $g(C')=t-1$ and $g(C'')=g-t-1\}$
of dimension $3g-4$.
\end{itemize}


  This is an example of a `description' list.

\begin{description}
\item[Zero case] $\rho(\Phi) = \{0\}$.

\item[Rational case] $\rho(\Phi) \ne \{0\}$ and $\rho(\Phi)$ is
contained in a line through $0$ with rational slope.

\item[Irrational case] $\rho(\Phi) \ne \{0\}$ and $\rho(\Phi)$ is
contained in a line through $0$ with irrational slope.
\end{description}
% \specialsection*{This is a Special Section Head}
% This is an example of a special section head%
%%%%%%%%%%%%%%%%%%%%%%%%%%%%%%%%%%%%%%%%%%%%%%%%%%%%%%%%%%%%%%%%%%%%%%%%
\footnote{Here is an example of a footnote. Notice that this footnote
text is running on so that it can stand as an example of how a footnote
with separate paragraphs should be written.
\par
And here is the beginning of the second paragraph.}%
%%%%%%%%%%%%%%%%%%%%%%%%%%%%%%%%%%%%%%%%%%%%%%%%%%%%%%%%%%%%%%%%%%%%%%%%
.

% \section{This is a numbered first-level section head}
% This is an example of a numbered first-level heading.

% \subsection{This is a numbered second-level section head}
% This is an example of a numbered second-level heading.

% \subsection*{This is an unnumbered second-level section head}
% This is an example of an unnumbered second-level heading.

% \subsubsection{This is a numbered third-level section head}
% This is an example of a numbered third-level heading.

% \subsubsection*{This is an unnumbered third-level section head}
% This is an example of an unnumbered third-level heading.

\begin{lemma}
Let $f, g\in  A(X)$ and let $E$, $F$ be cozero sets in $X$.
\begin{enumerate}
\item If $f$ is $E$-regular and $F\subseteq E$, then $f$ is $F$-regular.

\item If $f$ is $E$-regular and $F$-regular, then $f$ is $E\cup
F$-regular.

\item If $f(x)\ge c>0$ for all $x\in E$, then $f$ is $E$-regular.

\end{enumerate}
\end{lemma}

The following is an example of a proof.

\begin{proof} Set $j(\nu)=\max(I\backslash a(\nu))-1$. Then we have
\[
\sum_{i\notin a(\nu)}t_i\sim t_{j(\nu)+1}
  =\prod^{j(\nu)}_{j=0}(t_{j+1}/t_j).
\]
Hence we have
\begin{equation}
\begin{split}
\prod_\nu\biggl(\sum_{i\notin
  a(\nu)}t_i\biggr)^{\abs{a(\nu-1)}-\abs{a(\nu)}}
&\sim\prod_\nu\prod^{j(\nu)}_{j=0}
  (t_{j+1}/t_j)^{\abs{a(\nu-1)}-\abs{a(\nu)}}\\
&=\prod_{j\ge 0}(t_{j+1}/t_j)^{
  \sum_{j(\nu)\ge j}(\abs{a(\nu-1)}-\abs{a(\nu)})}.
\end{split}
\end{equation}
By definition, we have $a(\nu(j))\supset c(j)$. Hence, $\abs{c(j)}=n-j$
implies (5.4). If $c(j)\notin a$, $a(\nu(j))c(j)$ and hence
we have (5.5).
\end{proof}

\begin{quotation}
This is an example of an `extract'. The magnetization $M_0$ of the Ising
model is related to the local state probability $P(a):M_0=P(1)-P(-1)$.
The equivalences are shown in Table~\ref{eqtable}.
\end{quotation}

\begin{table}[ht]
\caption{}\label{eqtable}
\renewcommand\arraystretch{1.5}
\noindent\[
\begin{array}{|c|c|c|}
\hline
&{-\infty}&{+\infty}\\
\hline
{f_+(x,k)}&e^{\sqrt{-1}kx}+s_{12}(k)e^{-\sqrt{-1}kx}&s_{11}(k)e^
{\sqrt{-1}kx}\\
\hline
{f_-(x,k)}&s_{22}(k)e^{-\sqrt{-1}kx}&e^{-\sqrt{-1}kx}+s_{21}(k)e^{\sqrt
{-1}kx}\\
\hline
\end{array}
\]
\end{table}

\begin{definition}
This is an example of a `definition' element.
For $f\in A(X)$, we define
\begin{equation}
\mathcal{Z} (f)=\{E\in Z[X]: \text{$f$ is $E^c$-regular}\}.
\end{equation}
\end{definition}

\begin{remark}
This is an example of a `remark' element.
For $f\in A(X)$, we define
\begin{equation}
\mathcal{Z} (f)=\{E\in Z[X]: \text{$f$ is $E^c$-regular}\}.
\end{equation}
\end{remark}

\begin{example}
This is an example of an `example' element.
For $f\in A(X)$, we define
\begin{equation}
\mathcal{Z} (f)=\{E\in Z[X]: \text{$f$ is $E^c$-regular}\}.
\end{equation}
\end{example}

% \begin{xca}
% This is an example of the \texttt{xca} environment. This environment is
% used for exercises which occur within a section.
% \end{xca}

Some extra text before the \texttt{xcb} head. The \texttt{xcb} environment
is used for exercises that occur at the end of a chapter.  Here it contains
an example of a numbered list.

Here is an example of a cite. See \cite{Aoki}. \cite{Ao}

\begin{theorem}
This is an example of a theorem.
\end{theorem}

\begin{theorem}[Marcus Theorem]
This is an example of a theorem with a parenthetical note in the
heading.
\end{theorem}

\begin{figure}[tb]
\blankbox{.6\columnwidth}{5pc}
\caption{This is an example of a figure caption with text.}
\label{firstfig}
\end{figure}

\begin{figure}[tb]
\blankbox{.75\columnwidth}{3pc}
\caption{}\label{otherfig}
\end{figure}





\begin{xcb}{Exercises}
  \noindent Unless the contrary is explicitly stated, all numbers that are mentioned in these exercises are understood to be real.
  \begin{enumerate}
    \item  If $r$ is rational ($r \notin 0$) and $x$ is irrational, prove that $r + x$ and $rx$ are irrational.
    \item Prove that there is no rational number whose square is $12$.
    \item Prove Proposition \ref{some section}.
    \item  Let $E$ be a nonempty subset of an ordered set; suppose $\alpha$ is a lower bound of $E$ and $\beta$ 
    is an upper bound of $E$. Prove that $\alpha \leq \beta$.
    \item Let $A$ be a nonempty set of real numbers which is bounded below. Let $-A$ be the set of all numbers 
    $-x$, where $x \in A$. Prove that
    \begin{equation*}
      \inf A = -\sup (-A)
    \end{equation*}
    \item Fix $b>1$,
    \begin{enumerate}
      \item If $m,n,p,q$ are integers, $n>0,q>0,$ and $r=m/n=p/q$, prove that
      \begin{equation*}
        (b^m)^{l/n} =(b^p)^{l/q}.
      \end{equation*}
      Hence it makes sense to define $b^r = (b^m)^{1/n}$·
      \item Prove that $b^{r+s}= b^rb^s$ if $r$ and $s$ are rational.
      \item If $x$ is real, define $B(x)$ to be the set of all numbers $b^t$, where $t$ is rational and $t \leq x$. Prove that
      \begin{equation*}
        b^r = \sup B(r)
      \end{equation*}
      when $r$ is rational. Hence it makes sense to define
      \begin{equation*}
        b^x = \sup B(x)
      \end{equation*}
      for every real $x$.
      \item Prove that $b^{x+y} = b^xb^y$ for all real $x$ and $y$.
    \end{enumerate}
    \item satisfied
    \item Prove that no order can be defined in the complex field that turns it into an ordered field. Hint: $-1$ is a square.
    \item  Suppose $z=a+bi$, $w=c+di$. Define $z<w$ if $a < c$, and also if $a = c$ but $b < d$. Prove that this turns the set of all 
    complex numbers into an ordered set. (This type of order relation is ca11ed a \emph{dictionary order}, or \emph{lexicographic order}, for 
    obvious reasons.) Does this ordered set have the least-upper-bound property?
    \item Suppose $z=a+bi$, $w=u+vi$, and 
    \begin{equation*}
      a = \biggl(\frac{\left|w\right|+u}{2}\biggr)^{\nicefrac{1}{2}}, \qquad b = \biggl(\frac{\left|w\right|-u}{2}\biggr)^{\nicefrac{1}{2}}.
    \end{equation*}
    Prive that $z^2=w$ if $v\geq 0$ and that $(\overline{z})^2=w$ if $v\leq 0$. Conclude that every complex number (with one exception!) has two
    complex square roots.
    \item If $z$ is a complex number, prove that there exist an $r \geq 0$ and a complex number $w$ with $\left| w \right| = 1$ such that $z=rw$.
    Are $w$ and $r$ always uniquely determined by $z$?
    \item if $z_1,\cdots,z_n$ are complex, prove that
    \begin{equation*}
      \left|z_1+z_2+\cdots+z_n\right|\leq \left|z_1\right|+\left|z_2\right|+\cdots+\left|z_n\right|.
    \end{equation*}
    \item If $x,y$ are complex, prove that 
    \begin{equation*}
      \left|\left|x\right|-\left|y\right|\right|\leq \left|x-y\right|.
    \end{equation*}
    \item If $z$ is a complex number such that $\left|z\right| = 1$, that is, such that $z\overline{z} =1$, compute
    \begin{equation*}
      \left|1+z\right|^2+\left|1-z\right|^2.
    \end{equation*}
    \item Under what conditions does equality hold in the Schwarz inequality?
    \item Suppose $k \geq 3$, $\boldsymbol{x},\boldsymbol{y} \in \mathbb{R}^k$, $\left|x-y\right|=d>0$, and $r>0$. Prove:
    \begin{enumerate}
      \item If $2r > d$, there are infinitely many $\boldsymbol{z} \in \mathbb{R}^k$ such that
      \begin{equation*}
        \left|\boldsymbol{z}-\boldsymbol{x}\right|=\left|\boldsymbol{z}-\boldsymbol{y}\right|=r.
      \end{equation*}
      \item If $2r=d$, there is exactly one such $\boldsymbol{z}$.
      \item If $2r<d$, there is no such $\boldsymbol{z}$.\\
      How must these statements be modified if $k$ is $2$ or $1$?
    \end{enumerate}
    \item Prove that
    \begin{equation*}
      \left|\boldsymbol{x}+\boldsymbol{y}\right|^2+\left|\boldsymbol{x}-\boldsymbol{y}\right|^2=2\left|\boldsymbol{x}\right|^2+2\left|\boldsymbol{y}\right|^2
    \end{equation*}
    if $\boldsymbol{x}\in \mathbb{R}^k$ and $\boldsymbol{y}\in \mathbb{R}^k$. Interpret this geometrically, as a statement about parallelograms. 
    \item If $k \geq 2$ and $\boldsymbol{x} \in \mathbb{R}^k$, prove that there exists $\boldsymbol{y} \in \mathbb{R}^k$ such that $\boldsymbol{y} \neq \boldsymbol{0}$
    but $\boldsymbol{x} \cdot \boldsymbol{y}=0$. Is this also true if $k=1$?
    \item Supposea $\boldsymbol{a} \in \mathbb{R}^k,\boldsymbol{b} \in \mathbb{R}^k$. Find $\boldsymbol{c} \in \mathbb{R}^k$ and $r> 0$ such 
    that 
    \begin{equation*}
      \left|\boldsymbol{x}-\boldsymbol{a}\right|=2\left|\boldsymbol{x}-\boldsymbol{b}\right|
    \end{equation*}
    if and only if $\left|\boldsymbol{x}-\boldsymbol{c}\right|=r$.\\
    (\emph{Solution}: $3\boldsymbol{c}=4\boldsymbol{b}-\boldsymbol{a},3r=2\left|\boldsymbol{b}-\boldsymbol{a}\right|)$
    \item With reference to the Appendix, suppose that property \ref{some property} were omitted from the definition of a cut. 
    Keep the same definitions of order and addition. Show that the resulting ordered set has the least-upper-bound property, 
    that addition satisfies axioms \ref{first axiom} to \ref{end axiom} (with a slightly different zero-element!) but that \ref{some axiom} fails.
  \end{enumerate}
\end{xcb}


\ref{eq:test}


\endinput

%-----------------------------------------------------------------------
% End of chapter_01.tex
%-----------------------------------------------------------------------
