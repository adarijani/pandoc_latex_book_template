%-----------------------------------------------------------------------------
% Beginning of preface.tex
%-----------------------------------------------------------------------------
%
% AMS-LaTeX 1.2 sample file for a monograph, based on amsbook.cls.
% This is a data file input by chapter.tex.
%%%%%%%%%%%%%%%%%%%%%%%%%%%%%%%%%%%%%%%%%%%%%%%%%%%%%%%%%%%%%%%%%%%%%%%%

\chapter*{Preface}

\noindent This book is intended to serve as a text for the course in analysis that is usually taken by advanced undergraduates or by 
first-year students who study mathematics.\\
\indent The present edition covers essentially the same topics as the second one, with some additions, a few minor omissions, 
and considerable rearrangement. I hope that these changes will make the material more accessible and more attractive to the 
students who take such a course.\\
\indent Experience has convinced me that it is pedagogically unsound (though logically correct) to start off with the construction 
of the real numbers from the rational ones. At the beginning, most students simply fail to appreciate the need for doing this. 
Accordingly, the real number system is introduced as an ordered field with the least-upper-bound property, and a few interesting 
applications of this property are quickly made. However, Dedekind's construction is not omitted. It is now in an Appendix to Chapter 1, 
where it may be studied and enjoyed whenever the time seems ripe.\\
\indent The material on functions of several variables is almost completely rewritten, with many details filled in, and with 
more examples and more motivation. The proof of the inverse function theorem--the key item in Chapter 9--is simplified by means 
of the fixed point theorem about contraction mappings. Differential forms are discussed in much greater detail. Several applications 
of Stokes' theorem are included.\\
\indent As regard a other changes, the chapter on the Riemann-Stieltjes integral has been trimmed a bit, a short do-it-yourself
section on the gamma function has been added to Chapter 8, and there is a large number of new exercises, most of them with fairly
detailed hints.\\
\indent I have also included several references to articles appearing  in the \emph{American Mathematical Monthly} and in 
\emph{Mathematics Magazine}, in the hope that students will develop the habit of looking into the journal literature. Most of 
these references were kindly supplied by R. B. Burckel.\\
\indent Over the years, many people, students as well as teachers, have sent me corrections, criticisms, and other comments 
concerning the previous editions of this book. I have appreciated these, and I take this opportunity to express my sincere thanks 
to all who have written me.   

\aufm{WALTER RUDIN}

%-----------------------------------------------------------------------------
% End of preface.tex
%-----------------------------------------------------------------------------
